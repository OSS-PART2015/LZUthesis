\documentclass{book}

\usepackage{ctex}
%参考文献
%\usepackage[bibstyle=custom-numeric-comp]{biblatex}
%\usepackage[backend=biber,gbtverbose=true,
	%bibstyle=gbt7714_2005_n,citestyle=gbt7714_2005_n]{biblatex}
\usepackage[backend=biber,gbtverbose=true,
	bibstyle=gbt7714_2005_n,citestyle=gbt7714_2005_n]{biblatex}
%\usepackage[hyperref]{biblatex}
\addbibresource{ref.bib}
\renewcommand{\bibfont}{\zihao{5}}
\usepackage{hyperref} % Required for hyperlinks

%注意,这里一定要两个大括号,里面的那个大括号用于长标题在封面中的断行
\title{{兰州大学本科论文非官方 \LaTeX 模板}}
\author{沈周}

%\hypersetup{hidelinks,breaklinks=true,bookmarksopen=false,pdftitle={Title},pdfauthor={Author}}

\begin{document}
\tableofcontents
\mainmatter
\chapter{简介Introduction}
这是作者在2015年8月借学习《\citetitle{latextutorial}》\hyperref[latextutorial]{\cite{latextutorial}}一书之机,也为来年毕业论文之备写的一份{\it 非官方}模板。
\paragraph{jj}再到家庭,个人,都不可能是一帆风顺的,必然会有问题的出现,然而一旦问题出现,国家,个人的第一反应永远都不是解决问题,而是逃避和对责任的各种推卸。
前几天看腾讯推送的一个新闻:大量医疗垃圾所做成的一次性塑料饭盒,一次性水杯流入市场,危害极大。面对这样的事情,媒体的舆论导向不是鼓励“有关部门”对此类事件进行彻查,并且严肃处理,而是劝导消费者要保持警醒,时刻注意。我注意你妈了个鸡,我特么又不是火眼金睛,我肉眼凡胎的我知道所用的产品是什么做成的吗。这类产品,只要有人生产,流入市场,就一定会拥有消费人群,这类事件,如果源头还一直处在“我什么都不怕;我上面

\backmatter
\printbibliography[title={参考文献},heading=bibintoc]
\end{document}
