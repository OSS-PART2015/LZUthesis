\documentclass{LZU}

%参考文献
\usepackage[backend=biber,gbtverbose=true,
	bibstyle=gbt7714_2005_n,citestyle=gbt7714_2005_n]{biblatex}
\addbibresource{ref.bib}
\renewcommand{\bibfont}{\zihao{5}}

%注意,这里一定要两个大括号,里面的那个大括号用于长标题在封面中的断行
\title{{兰州大学本科论文非官方 \LaTeX 模板}}
\entitle{{The unofficial \LaTeX{} template for }{the undergraduate thesis of Lanzhou University}}

%package
\usepackage{minted} %代码格式
\usepackage[nameinlink]{cleveref}
\usepackage{dirtree}
\usepackage{siunitx}

\newcommand{\filename}[1]{{\ttfamily #1}}
\begin{document}
\maketitle
\makestatement
\frontmatter
\ZhAbstract{你好,这个论文的\LaTeX 模板啊,是我根据论文的要求自己写的,凑活着用呗。好像还是不够长,再写两句。写什么呢。}{你好;好的}
\EnAbstract{
As the first command of the paragraph. This might come in handy when you start a document with body text and not with a sectioning command.

Be careful, however, if you decide to set the indent to zero, then it means you will need a vertical space between paragraphs in order to make them clear. The space between paragraphs is held in , which could be altered in a similar fashion as above. However, this parameter is used elsewhere too, such as in lists, which means you run the risk of making various parts of your document look very untidy by changing this setting. If you want to use the style of having no indentation with a space between paragraphs, use the parskip package, which does this for you, while making adjustments to the spacing of lists and other structures which use paragraph spacing, so they don't get too far apart. If you want both indent and break, use
}{hello, world}
\tableofcontents
\mainmatter
\chapter{简介}
这是作者在2015年8月借学习《\LaTeX 入门》\cite{latextutorial}一书之机,也为来年毕业论文之备写的一份{\it 非官方}模板。
\chapter{模板使用}
\section{你好,世界}
\subsection{一个最简单的例子}
首先,我们给出使用本模板的一个最简单的例子,见代码清单\ref{lst:simplest}。

\begingroup
    \captionof{listing}{此模板的一个最简单的例子}
    \label{lst:simplest}
    \inputminted[breaklines,frame=single,linenos]{latex}{simplest.tex}
\endgroup
此代码清单\ref{lst:simplest}保存在\filename{simplest.tex}中,可以安以下过程编译:
\begin{minted}{bash}
    xelatex simplest.tex
    biber simplest
    xelatex simplest.tex
    xelatex simplest.tex
\end{minted}
在Linux系统中,可以直接输入
\begin{minted}{bash}
    make simplest
\end{minted}
或者在Windows系统中,运行\filename{compile.bat}以自动完成上述过程。

\subsection{信息配置}
如果成功编译,可以看到在封面页,有一些信息在主文件\filename{simplest.tex}中并没有出现,如导师姓名,学生所属学院等。这些信息的修改在\filename{LZU.cfg}中形如
\begin{minted}{latex}
    \def\LZU@value@***{***}
\end{minted}
这样的行。
\section{文件结构}
\dirtree{%
    .1 ./.
    .2 template.cls\DTcomment{主文件}.
    .2 LZU.cls\DTcomment{cls模板文件}.
    .2 LZU.cfg\DTcomment{配置文件}.
    .2 pic/.
    .3 lzu.eps\DTcomment{校名图片}.
    .2 Makefile\DTcomment{Linux自动编译脚本}.
    .2 compile.bat\DTcomment{Windows自动编译脚本}.
    .2 gbt7714\_2005.def\DTcomment{参考文献格式配置}.
    .2 gbt7714\_2005\_n.bbx\DTcomment{参考文献格式配置}.
    .2 gbt7714\_2005\_n.cbx\DTcomment{参考文献格式配置}.
}
\chapter{格式说明}
毕业论文用 A4 标准纸($\SI{210}{mm}\times \SI{297}{mm}$)打印、印刷或复印,按论文顺序装订成册,论文顺序依次为:封面(包括扉页)、诚信责任书、关于毕业论文(设计)使用授权的申明、中文摘要、英文摘要、目录、论文正文、参考文献、附录、致谢、评语。论文页边距一般要求:上边距 \SI{3}{cm}、下边距\SI{2.54}{cm},左右边距\SI{3.17}{cm},页眉页脚\SI{2.0}{cm}。
\section{封面}
论文封面颜色:本科生毕业论文封面统一为白色。

论文题目用三号字,宋体,加粗,其他信息用小三号字,宋体, 加粗,居中。
\section{正文(Contents)}
\subsection{标题}
\begin{itemize}
    \item 正文标题:一级标题为三号字,黑体,加粗,居中,单倍行距,段前 24 磅,段后 18 磅;
    \item 二级标题为四号字,黑体,顶左,单倍行 距,段前 24 磅,段后 6 磅;
    \item 三级标题为小四号字,黑体,首行缩进2个汉字符,单倍行距,段前12磅,段后6磅。正文:采用小四号字,宋体(英文用 Times New Roman 体,12磅),两端对齐,段落首行左缩进2个汉字符,行距20磅,段前段后0磅。
\end{itemize}

\subsection{图表}
\subsubsection{图}
图名置于图的下方,五号字,宋体,居中,单倍行距,段前 6 磅,段后 12 磅,图序与图名之间空 1 个汉字符。
\subsubsection{表}
表名置于表的上方,五号字,宋体,居中,单倍行距,段 前 6 磅,段后 6 磅,表序与表名之间空 1 个汉字符。表下方的注释 为五号字,宋体,居左(英文用 Times New Roman 体 10.5 磅),单倍行距。
\subsubsection{注释}
一般分为页末注(脚注)和篇末注。脚注,宋体, 9 磅(英 文用 Times New Roman,9 磅),左对齐,单倍行距,段前段后 0 磅, 按阿拉伯数字编号,每页须重新编号。
\subsection{参考文献}
参考文献是文中引用的有具体文字来源的文献集合, 毕业论文中引用他人成果之处均应如实、详细地列出参考文献目录。各种主要参考文献按如下格式编排:
\begin{itemize}
    \item 专著、论文集、学位论文、报告:[序号]主要责任者.文献题 名[文献类型标识M/C/D/R].出版地:出版者,出版年.起止页码(任选).
    \item 学术期刊:[序号]主要责任者.文献题名[J].刊名,年,卷 (期):起止页码.
    \item 报纸文章:[序号]主要责任者.文献题名 [N].报纸名,出版日期(版次).
    \item 专利:[序号]专利所有者.专利题名[P].专利国别:专利号,授权日期.
    \item 技术标准:[序号]标准编号,标准名称[S].
    \item 电子文献:[序号]主要责任者.电子文献题名[电子文献和载体类型标识].电子文献的出处或可获得地址,发表或更新日期/引用日期(任选).
\end{itemize}

\printbibliography[title={参考文献},heading=bibintoc]
\Appendix
图图图图图图图图图图图图图图图图图图图图图图图图图图图图图图图图图图图图图图图图图图图图图图图图图图图图图图图图图图图图图图图图图图图图图图图图图图图图图图图图图图图图图图图图图图图图图图图图图图图图图\footnote{注释}
\Thanks
感谢戴维同学的测试。
\Grade
\end{document}
